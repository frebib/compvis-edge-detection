\documentclass[a4paper]{article}

\usepackage[margin=1.9cm]{geometry}
\usepackage[none]{hyphenat}
\usepackage{multicol, caption}
\usepackage{titlesec}
\usepackage{graphicx}
\usepackage[export]{adjustbox}

\graphicspath{ {test-images/} }
\usepackage{csquotes}
\usepackage[UKenglish,USenglish]{babel}

\usepackage[dvipsnames]{xcolor}

\usepackage{pgfplots}
\pgfplotsset{compat=newest}

\newsavebox{\measuredSize}
\newcommand{\resizeToWidth}[2]{%
    \pgfmathsetmacro{\pgfplotswidth}{#2}%
    \begin{lrbox}{\measuredSize}#1\end{lrbox}%
    \pgfmathsetmacro{\pgfplotswidth}{2*\pgfplotswidth-\wd\measuredSize}%
    #1%
}

\usepackage{etoolbox}
\newskip\mfilskip
\mfilskip=0pt plus 3cm\relax
\newcommand{\mfilbreak}{\vspace{\mfilskip}\penalty -200%
  \ifdim\lastskip<\mfilskip\vspace{-\lastskip}\else\vspace{-\mfilskip}\fi}
\pretocmd{\section}{\mfilbreak}{}{}

\usepackage{enumitem}
\setlist{noitemsep}

\newenvironment{floatfig}
  {\par\medskip\noindent\minipage{\linewidth}}
  {\endminipage\par\medskip}

\titleformat*{\section}{\large\bfseries}
\titleformat*{\subsection}{\normalsize\bfseries}
\titlespacing*\section{0pt}{12pt plus 4pt minus 2pt}{4pt}
\titlespacing*\section{0pt}{10pt plus 3pt minus 2pt}{3pt}

\usepackage[authoryear,sort,square,longnamesfirst]{natbib}

\begin{document}

\noindent{\Huge Edge detection methods for the identification of HEp-2 cell immunofluorescence in detecting\linebreak anti-nuclear antibodies}
\vspace{1em}

\noindent{\large 1146273 \qquad 1467414}
\vspace{.5em}

\begin{multicols*}{2}

\section*{\normalsize Abstract}

\textbf{\small Serological testing for anti-nuclear antibodies by immunofluorescence is~yet to be standardised and remains a subjective exercise liable to high inter- and intra-laboratory variation. The use of digital imaging techniques in the evaluative process can improve objectivity by entrusting algorithms with the task of discerning edges in the fluoresced image. To address this hypothesis we investigated the efficacy of various edge detectors known to the field of computational vision research. Application of ROC analysis to the results generated quantitative feedback regarding the efficacy of each edge detection method, suggesting METHOD X to be the most concrete standard by which to detect immunofluorescence.}

\section*{Introduction}

The purpose of this experiment was to objectively assess the efficacy of different edge detectors used in immunofluorescence imaging. The methods investigated were; simple gradient, Roberts, Sobel, first order Gaussian, Laplacian and the Laplacian of Gaussian. Application of thresholding in post- processing was used to evaluate the efficacy of each method by comparing the edge points identified with those present in the pre-edge-detected images which had been provided in the laboratory environment. ROC analysis was used to determine both the sensitivity and specificity of each edge detection method. 

Edge detection looks for changes in the intensity (i.e. brightness) of an image. The methods can be divided into two general groups: those which search for changes in the first derivative (eg gradient magnitude) and those which search for zero-crossings in the second derivative. 

Convolving the original image with a mask (also called a ‘filter’) will identify the rate of change for a given pixel, and this must be done in both the x and y axes. The magnitude (or an approximation thereof) is then calculated with respect to the other pixels in the image. Commonly a noise filtering method will be applied in advance of edge detection, giving an exaggerated or more clearly weighted numerical representation of an image which is therefore said to contain less ‘noise’. Examples of noise filtering include a linear filter such as a mean filter or Gaussian filtering.

\section*{Method and Materials}

We were provided with three bitmap images of fluorescing cells, along with separate images of their manually detected edges. These were to act as our target edge-detection results.

The (original) images were first converted to grayscale in order to maximise their intensity contrast and provide us with intensity images (data matrices).This simplified the task of edge detection (i.e. identifying changes of intensity in the image) as it made the various strengths of edges more distinct. The images were then inverted to leave black cells on a white background. 

\section*{Noise reduction}

Gaussian smoothing was then applied to the grayscale images as a method of noise reduction. This is a low-pass filtering technique and so suppresses high-frequency detail (i.e. noise, but also, potentially, edges) whilst preserving the low frequency parts of the image. This served to reduce the chances that a method of edge detection would be compromised by random or non-essential details present in the images.

We applied two one-dimensional Gaussian filters in succession, as opposed to a single two-dimensional filter, for reasons of computational efficiency. Using filters of greater dimensions acted to preserve the local area for averaging purposes. For comparison, we also applied each filter in two passes in order to compare the effectiveness of increased noise suppression against a selection of detectors.

The edge detectors used in this test are:

\vspace{-0.3em}

\begin{multicols*}{2}
    \begin{itemize}[noitemsep]
        \item Gradient
        \item Sobel
        \item Roberts
        \item Laplacian
        \item First Derivative Gaussian
        \item Canny
    \end{itemize}
\end{multicols*}

Each edge detector was applied in turn to each low-pass filtered image to produce the results seen in Figure 1.

\vspace{.6em}
\noindent
\begin{floatfig}
    \resizeToWidth{
        \begin{tikzpicture}
            \begin{axis}[
                xlabel={1 - Specificity \% [FPR]},
                ylabel={Sensitivity \% [TPR]},
                width=\pgfplotswidth * 0.9,
                xmin=0, xmax=100,
                ymin=0, ymax=100,
                grid style=dashed,
                yticklabel style={
                    align=right,
                    inner sep=0,
                    xshift=-0.3em,
                }
            ]

            \addplot+[
                    only marks, color=Blue, mark=square,
                    x filter/.code={\pgfmathparse{(1-#1)*100}\pgfmathresult},
                    y filter/.code={\pgfmathparse{#1*100}\pgfmathresult}
                ]
                coordinates {
                    % No Gaussian Filter
                    (0.5854, 0.9873) % Simple Gradient
                    (0.2199, 0.9994) % Sobel
                    (0.7144, 0.9698) % Roberts
                    (0.8006, 0.4071) % Laplacian
                    (0.7844, 0.9426) % F-O-Gaussian
                    (0.8443, 0.3078) % Canny
                };

            \addplot+[
                    only marks, color=Cyan, mark=square,
                    x filter/.code={\pgfmathparse{(1-#1)*100}\pgfmathresult},
                    y filter/.code={\pgfmathparse{#1*100}\pgfmathresult}
                ]
                coordinates {
                    % 3x3 Gaussian, 1-Pass
                    (0.6743, 0.9852) % Simple Gradient
                    (0.2962, 0.9993) % Sobel
                    (0.7432, 0.9695) % Roberts
                    (0.8952, 0.3474) % Laplacian
                    (0.7863, 0.9369) % F-O-Gaussian
                    (0.8550, 0.3049) % Canny
                };

            \addplot+[
                    only marks, color=Magenta, mark=square,
                    x filter/.code={\pgfmathparse{(1-#1)*100}\pgfmathresult},
                    y filter/.code={\pgfmathparse{#1*100}\pgfmathresult}
                ]
                coordinates {
                    % 3x3 Gaussian, 2-Pass
                    (0.6993, 0.9842) % Simple Gradient
                    (0.3463, 0.9991) % Sobel
                    (0.7523, 0.9670) % Roberts
                    (0.9277, 0.2879) % Laplacian
                    (0.7875, 0.9311) % F-O-Gaussian
                    (0.8650, 0.3018) % Canny
                };

            \addplot+[
                    only marks, color=Peach, mark=square,
                    x filter/.code={\pgfmathparse{(1-#1)*100}\pgfmathresult},
                    y filter/.code={\pgfmathparse{#1*100}\pgfmathresult}
                ]
                coordinates {
                    % 5x5 Gaussian, 1-Pass
                    (0.7138, 0.9782) % Simple Gradient
                    (0.4164, 0.9986) % Sobel
                    (0.7568, 0.9564) % Roberts
                    (0.9629, 0.1558) % Laplacian
                    (0.7901, 0.9152) % F-O-Gaussian
                    (0.8856, 0.2931) % Canny
                };

            \addplot+[
                    only marks, color=RedOrange, mark=square,
                    x filter/.code={\pgfmathparse{(1-#1)*100}\pgfmathresult},
                    y filter/.code={\pgfmathparse{#1*100}\pgfmathresult}
                ]
                coordinates {
                    % 5x5 Gaussian, 2-Pass
                    (0.7116, 0.9782) % Simple Gradient
                    (0.4432, 0.9979) % Sobel
                    (0.7561, 0.9372) % Roberts
                    (0.9866, 0.0544) % Laplacian
                    (0.7942, 0.8892) % F-O-Gaussian
                    (0.9109, 0.2800) % Canny
                };

            % Simple Gradient Line
            \addplot+[
                    color=black, mark=none, solid,
                    x filter/.code={\pgfmathparse{(1-#1)*100}\pgfmathresult},
                    y filter/.code={\pgfmathparse{#1*100}\pgfmathresult}
                ]
                coordinates {
                    (0.5854, 0.9873) % No Gaussian Filter
                    (0.6743, 0.9852) % 3x3 Gaussian, 1-Pass
                    (0.6993, 0.9842) % 3x3 Gaussian, 2-Pass
                    (0.7138, 0.9782) % 5x5 Gaussian, 1-Pass
                    (0.7116, 0.9782) % 5x5 Gaussian, 2-Pass
                };

            % Sobel Line
            \addplot+[
                    color=black, mark=none, solid,
                    x filter/.code={\pgfmathparse{(1-#1)*100}\pgfmathresult},
                    y filter/.code={\pgfmathparse{#1*100}\pgfmathresult}
                ]
                coordinates {
                    (0.2199, 0.9994) % No Gaussian Filter
                    (0.2962, 0.9993) % 3x3 Gaussian, 1-Pass
                    (0.3463, 0.9991) % 3x3 Gaussian, 2-Pass
                    (0.4164, 0.9986) % 5x5 Gaussian, 1-Pass
                    (0.4432, 0.9979) % 5x5 Gaussian, 2-Pass
                };

            % Roberts Line
            \addplot+[
                    color=black, mark=none, solid,
                    x filter/.code={\pgfmathparse{(1-#1)*100}\pgfmathresult},
                    y filter/.code={\pgfmathparse{#1*100}\pgfmathresult}
                ]
                coordinates {
                    (0.7144, 0.9698) % No Gaussian Filter
                    (0.7432, 0.9695) % 3x3 Gaussian, 1-Pass
                    (0.7523, 0.9670) % 3x3 Gaussian, 2-Pass
                    (0.7568, 0.9564) % 5x5 Gaussian, 1-Pass
                    (0.7561, 0.9372) % 5x5 Gaussian, 2-Pass
                };

            % Laplacian Line
            \addplot+[
                    color=black, mark=none, solid,
                    x filter/.code={\pgfmathparse{(1-#1)*100}\pgfmathresult},
                    y filter/.code={\pgfmathparse{#1*100}\pgfmathresult}
                ]
                coordinates {
                    (0.8006, 0.4071) % No Gaussian Filter
                    (0.8952, 0.3474) % 3x3 Gaussian, 1-Pass
                    (0.9277, 0.2879) % 3x3 Gaussian, 2-Pass
                    (0.9629, 0.1558) % 5x5 Gaussian, 1-Pass
                    (0.9866, 0.0544) % 5x5 Gaussian, 2-Pass
                };

            % F-O-Gaussian Line
            \addplot+[
                    color=black, mark=none, solid,
                    x filter/.code={\pgfmathparse{(1-#1)*100}\pgfmathresult},
                    y filter/.code={\pgfmathparse{#1*100}\pgfmathresult}
                ]
                coordinates {
                    (0.7844, 0.9426) % No Gaussian Filter
                    (0.7863, 0.9369) % 3x3 Gaussian, 1-Pass
                    (0.7875, 0.9311) % 3x3 Gaussian, 2-Pass
                    (0.7901, 0.9152) % 5x5 Gaussian, 1-Pass
                    (0.7942, 0.8892) % 5x5 Gaussian, 2-Pass
                };

            % Canny Line
            \addplot+[
                    color=black, mark=none, solid,
                    x filter/.code={\pgfmathparse{(1-#1)*100}\pgfmathresult},
                    y filter/.code={\pgfmathparse{#1*100}\pgfmathresult}
                ]
                coordinates {
                    (0.8443, 0.3078) % No Gaussian Filter
                    (0.8550, 0.3049) % 3x3 Gaussian, 1-Pass
                    (0.8650, 0.3018) % 3x3 Gaussian, 2-Pass
                    (0.8856, 0.2931) % 5x5 Gaussian, 1-Pass
                    (0.9109, 0.2800) % 5x5 Gaussian, 2-Pass
                };

            \addplot+[color=red, mark=x, mark size=5] coordinates { (17.44, 39.55) };

            %\legend{Gaussian 3x3 1-Pass, Gaussian 3x3 2-Pass, Gaussian 5x5 1-Pass, Gaussian 5x5 2-Pass}
            \draw[gray, dashed]
                (axis cs:\pgfkeysvalueof{/pgfplots/xmin},\pgfkeysvalueof{/pgfplots/xmin}) -- 
                (axis cs:\pgfkeysvalueof{/pgfplots/xmax},\pgfkeysvalueof{/pgfplots/xmax});
            \end{axis}
        \end{tikzpicture}
    }{\columnwidth}

    \vspace{-1em}
    \centering
        \captionof{figure}{\emph{Effects of noise filtering on edge detection\label{fig:filtergraph}}}
    \vspace{0.6em}
\end{floatfig}

\noindent It should be noted that whilst the ROC values may not show particularly ideal \emph{Sensitivity} or \emph{Specificity} values, this was not the aim at this stage of the experiment.  The aim here was to show the relative difference in \emph{noise reduction} techniques in advance of further configuring each detection method for each image. A fixed threshold of \textbf{0.01} or \textbf{1\%} was used here.

\section*{Thresholding}

Finally, thresholds were determined for the processed images. This was done on a case-by-case basis owing to the idiosyncrasies of each. The application of these thresholds equated to specifying which ‘edges’ were true edges (i.e. having values above the threshold) and so should be kept, and which were spurious and could be disregarded. At this stage we considered whether human evaluation of the efficacy of thresholding would be appropriate given the circumstances (i.e. carrying out eventual ROC analysis, a quantitative technique). An alternative would be to determine thresholds mathematically, for instance by hypothesising that a percentage of each image would comprise of edges and then ensuring that thresholding brought the resulting image within this range. From an analysis of the pre-edge detected images with which we had been provided we calculated the desired edge ‘content’ of the final images (post-edge detection) to be approximately 8\% (for ‘10905 JL Edges.bmp’, fig. 1), 9\% (for ‘43590 AM Edges.bmp’, fig. 2) and 10\% (for ‘9343 AM Edges.bmp’, fig. 3), respectively. We kept these approximations in mind and carried out thresholding appropriately throughout the experiment.

\section*{Results}

For image 9343 AM, the noisiest of the three, a sequence of noise-reduction filters was necessary to achieve optimal smoothing:

\begin{itemize}[noitemsep]
    \item No filter
    \item 3x3 Gaussian Filter (1 \& 2 Pass)
    \item 5x5 Gaussian Filter (1 \& 2 Pass)
\end{itemize}

\noindent Increasing the strength and/or repition(s) of any of the smoothing filters used affected the ROC analysis of every edge detector by increasing the specificity. Applicable to only a selection of the filter-detector combinations, an increased level of noise-reduction had an adverse effect on the sensitivity of the output image, most noticeably the \emph{Laplacian} edge detector when paired with the \emph{2-Pass 5x5 Gaussian} smoothing filter.

\vspace{.4em}
\begin{floatfig}
    \centering
        \adjincludegraphics[
            width=.9\textwidth,
            trim={{.2\width} {.32\height} {.2\width} {.32\height}},
            clip
        ]{diff-lapl-c1-gaus5-p2.png}
    \captionof{figure}{\emph{The difference-image of \textbf{9343 AM.bmp} applied with 2-Pass 5x5 Gaussian and detected by Laplacian\label{fig:diff}}}
\end{floatfig}

As can be seen from Figure~\ref{fig:diff}. of the actual result vs the 'target' image, there was little overlap between what was detected and what was actually discerned to be an edge. Anything represented by \emph{pink} is a real edge, not detected; whilst anything represented by \emph{green} was detected but is not, in fact, a real edge.

The probable main cause for this behaviour was the choice threshold. The same test conducted with a reduced threshold of only \textbf{2\% / 0.02} produced a closer to expected ROC value denoted by the \textcolor{red}{red X} in Figure~\ref{fig:filtergraph}.

\subsection*{Noise Reduction Conclusion}
We can conclude from the tests conducted that an increased smoothing filter size and/or repitition(s) can change the outcome of the ROC analysis, tending towards higher specificity but a lower sensitivity. This can be likened to traversing along the \emph{convex hull} towards (0, 0).

\section*{Edge detection}

The various edge detection filters were then convolved with the original images. This computed the weighted sum of pixels in the image, in both the x and y axes (directions).

The following curves are those of 3 variations of \emph{10905 JL} with  First-Derivative Gaussian detector

\vspace{-0.4em}
\begin{itemize}[noitemsep]
    \item No Smoothing - Cyan
    \item 1 Pass Gaussian 5x5 - Blue
    \item 2 Pass Gaussian 5x5 - Purple
\end{itemize}

Each curve represents the ROC values between thresholds of \textbf{0} and \textbf{0.1} or \textbf{10\%} reaching right from one corner on the \textit{line of no discrimination} to the other in a large arc.

\vspace{.6em}
\noindent
\begin{floatfig}
    \resizeToWidth{
        \begin{tikzpicture}
            \begin{axis}[
                xlabel={1 - Specificity \% [FPR]},
                ylabel={Sensitivity \% [TPR]},
                width=\pgfplotswidth * 0.9,
                xmin=0, xmax=25,
                ymin=75, ymax=100,
                grid style=dashed,
                yticklabel style={
                    align=right,
                    inner sep=0,
                    xshift=-0.3em,
                }
            ]

            \pgfplotstableread{test-images/cells2.txt}{\cellstwo}
            \pgfplotstableread{test-images/cells2g5.txt}{\cellstwogfive}
            \pgfplotstableread{test-images/cells2g5p2.txt}{\cellstwogfiveptwo}
            \addplot[
                cyan,
                x filter/.code={\pgfmathparse{(1-#1)*100}\pgfmathresult},
                y filter/.code={\pgfmathparse{#1*100}\pgfmathresult}
            ] table [x={x}, y={y}] {\cellstwo};
            \addplot[
                blue,
                x filter/.code={\pgfmathparse{(1-#1)*100}\pgfmathresult},
                y filter/.code={\pgfmathparse{#1*100}\pgfmathresult}
            ] table [x={x}, y={y}] {\cellstwogfive};
            \addplot[
                purple,
                x filter/.code={\pgfmathparse{(1-#1)*100}\pgfmathresult},
                y filter/.code={\pgfmathparse{#1*100}\pgfmathresult}
            ] table [x={x}, y={y}] {\cellstwogfiveptwo};
            \end{axis}
        \end{tikzpicture}
    }{\columnwidth}

    \vspace{-1em}
    \centering
        \captionof{figure}{\emph{Noise filtering on edge detection}\label{fig:filtergraph}}
    \vspace{0.6em}
\end{floatfig}

The closest point on the unsmoothed curve is with a threshold of \textbf{0.0215} or \textbf {2.15\%}. Surprisingly the single and double pass gaussian smoothed variants have a slightly lower specificity with the 2-pass producing a significantly worse result relative to the other two.


\section*{Discussion}

Issues of quantitative accuracy remain pressing regarding the imaging of bioluminescent and fluorescing systems \citep{taylor2015accounting}. 

The issues of how best to minimise false positives (i.e. noise) and preserve image resolution in fluorescence imaging have been explored by Dedecker et al. who are in favour of the ‘superresolution’ technique. This involves reversibly photoswitching fluorescing proteins and thereby generating a sequence of images over time. A superresolution image can then be extracted using quantitative analysis of the rate of change of fluorescence, promising an image of significantly higher resolution and with reduced noise \citep{dedecker2012widely}. This may have produced more favourable results in the case of '9343 AM.bmp' which exhibited significant background noise and loss of resolution post-processing.

Also of interest is bioluminescence tomography which sections the (3D) image and then reconstructs it. This is reported to produce significantly more stable (i.e. more consistent) luminescence, particularly with respect to the depth of the sample being processed, helping to accurately home-in on the source of luminescence. Given that we are concerned with serological testing, this may well be an important option to explore, opening up the possibility of 3D sampling. Bioluminescence tomography ‘…therefore has the potential to both increase the amount and the accuracy of quantitative data attained by luminescence imaging’ \citep{guggenheim2013bioluminescence}.

\bibliographystyle{plainnat}
\bibliography{refs}

\end{multicols*}

\end{document}
